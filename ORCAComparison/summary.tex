\documentclass[12pt]{article}
%--------------------   start of the 'preamble'
%
\usepackage{graphicx,amssymb,amstext,amsmath,color}
\usepackage[margin=2cm]{geometry}
\usepackage{abstract}
\usepackage{setspace}
\usepackage[footnotesize,bf]{caption}

% TABLE
\usepackage{multicol,hhline,colortbl,multirow}
\usepackage{braket}
\usepackage{siunitx}
\usepackage{hyperref}
\usepackage{authblk}
\usepackage{siunitx}
\usepackage{mathrsfs}
\usepackage[sort&compress]{natbib}
\bibpunct{[}{]}{,}{s}{}{;}


\definecolor{gray}{gray}{0.8}
\def\mobunits{\square\centi\meter\per\volt\per\second}
\def\gcm{\gram\per\cubic\centi\meter}
\def\ccg{\cellcolor{gray}}

\renewcommand{\labelitemii}{$\circ$}
\renewcommand{\bibname}{References}



\title{Comparison of P3HT Energy Levels from ORCA to DFT}
\author{Matthew Jones}
\date{\today}

\begin{document}
\maketitle

\section{Systems Studied}

We have randomly selected several pairs of chromophores from the ordered P3HT morphology (T1.5) to explore using both ORCA's ZINDO/S calculations, and Chris' more rigorous DFT methods.
The systems studied contained 50 atoms (2 monomers) from the following pairs of chromophores:
\begin{itemize}
    \item{0469-3714}
    \item{0841-1237}
    \item{2032-2900}
\end{itemize}

\section{Energy Levels}


\begin{center}
\begin{tabular}{| c | c | c |}
\hline
\rule{0pt}{2.5ex} 
\multirow{2}{*}{\textbf{Level}}&\textbf{DFT}&\textbf{ZINDO/S}\\
%\multicolumn{2}{| c |}{\textbf{DFT}}&\multicolumn{2}{| c |}{\textbf{ZINDO/S}}\\
&\textbf{eV}&\textbf{eV}\\
\hhline{|===|}
\multicolumn{3}{| c |}{\ccg\textbf{0469-3714}}\\
\hhline{---}
HOMO - 1&-6.508&-8.271\\
\ccg HOMO&\ccg -6.312&\ccg -8.176\\
LUMO&-0.695&0.085\\
\ccg LUMO + 1&\ccg -0.457&\ccg 0.230\\
Bandgap&5.617&8.091\\
\ccg HOMO Split&\ccg 0.196&\ccg 0.095\\
\hhline{---}
\multicolumn{3}{| c |}{\textbf{0841-1237}}\\
\hhline{---}
\ccg HOMO - 1&\ccg -6.600&\ccg -8.266\\
HOMO&-6.401&-8.208\\
\ccg LUMO&\ccg -0.714&\ccg 0.147\\
LUMO + 1&-0.502&0.222\\
\ccg Bandgap&\ccg 5.687&\ccg 8.061\\
HOMO Split&0.199&0.058\\
\hhline{---}
\multicolumn{3}{| c |}{\ccg \textbf{2032-2900}}\\
\hhline{---}
HOMO - 1&-6.578&-8.343\\
\ccg HOMO&\ccg -6.492&\ccg -8.335\\
LUMO&-0.727&0.004\\
\ccg LUMO + 1&\ccg -0.523&\ccg 0.169\\
Bandgap&5.765&8.331\\
\ccg HOMO Split&\ccg 0.086&\ccg 0.008\\
\hhline{---}
\end{tabular}\label{table:mob}
\captionof{table}{A comparison of the frontier molecular orbital energy levels, predicted bandgap, and dimer HOMO splitting from ZINDO/S and DFT for each examined chromophore pair.}
\end{center}

\begin{itemize}
    \item{The results don't look super promising.}
    \item{ZINDO consistently reports deeper HOMOs (by an average of 28\% corresponding to around 2 eV) and shallower LUMOs (by 11\%, around 0.7 eV) for each chromophore dimer pair.}
    \item{Since we scale the HOMO levels anyway and don't even touch the unoccupied orbitals for donor chromophores, this might not be a big problem - a more important comparison is the HOMO splitting value.}
    \item{\textbf{The ZINDO/S calculations consistently underpredict the HOMO splitting}, suggesting less significant orbital overlap and therefore slower charge transport.
        This is a bit of an issue, as we're already on the high-side of the experimental mobilities in our mobility calculations, so we can't really afford for it to be significantly faster.}
    \item{In the best case, ZINDO/S predicts the HOMO splitting correct to a factor of about 2, but in the worst case, it's an order of magnitude too low.}
    \item{The absolute deviation between the two calculations also varies by a factor of about 2, from 78 meV to 141 meV.
        These numbers correspond to the approximate binding energy of an exciton so, although we don't consider it, in a real device this could make the difference between an exciton being allowed to form and not.}
\end{itemize}


\clearpage

\bibliography{refs}
\bibliographystyle{unsrt}


\end{document}
